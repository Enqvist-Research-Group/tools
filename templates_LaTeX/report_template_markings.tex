\documentclass[12pt] {article}
\usepackage[margin=1in]{geometry}
\usepackage{graphicx}
\usepackage[numbers]{natbib} %Bibliography.
\usepackage{xcolor}
\usepackage{comment}
\usepackage{float}
\usepackage{multirow}
% \usepackage{authorblk}

\usepackage{hyperref}
\hypersetup{
    colorlinks=true, %set true if you want colored links
    linktoc=all,     %set to all if you want both sections and subsections linked
    citecolor=blue,
    filecolor=blue,
    linkcolor=blue,
    urlcolor=blue,
}


\usepackage{fancyhdr}
\usepackage{lastpage}

\usepackage{pdfpages}

% document body header/footer
\pagestyle{fancy}
\fancyhf{}
\lhead{body LA-UR-...}
\chead{body LA-UR-...}
\rhead{body LA-UR-...}
\lfoot{body LA-UR-...}
\cfoot{\thepage\ of \pageref{LastPage}}
\rfoot{body LA-UR-...}

% document title page header/footer setup
\fancypagestyle{plain}{
  \fancyhf{}
  \fancyhead[l]{titlepage LA-UR-...}
  \fancyhead[c]{titlepage LA-UR-...}
  \fancyhead[r]{titlepage LA-UR-...}
  \fancyfoot[c]{\thepage\ of \pageref{LastPage}}
  \renewcommand{\headrulewidth}{0.4pt} 
  \renewcommand{\footrulewidth}{0.4pt} 
}



%\rfoot{Handout 1}

%\setcitestyle{numbers} %Cite as numbers or author-year.
% \bibliographystyle{plain}


\begin{document}
\title{A title}
\author{F. Lastname}
\date{\today}

% \maketitle
% \thispagestype{empty}
% \clearpage
\maketitle
% \thispagestyle{empty}

\begin{abstract}
An abstract
\end{abstract}

\newpage

%\setcounter{tocdepth}{2}
\tableofcontents

\listoffigures

\listoftables


\newpage

% \chapter{start}
\section{Project Overview}
Citing nearly all BibTeX field types%  to show what in-line citations and bibliography will look liked (\cite{NIFneutronSources}, \cite{neutronGammaEnvForTheACRR}, \cite{ACRRdiagramSource}, \cite{wsmrPENTRANthesis}, \cite{ToxProfileForUranium}, \cite{TotalXSectionsForFourteenMeVn}, \cite{BCPotentiallyDangerous}, \cite{HighTempPolymers}, \cite{SputteredBCThinFilms}, \cite{MercuryUserGuide}, \cite{LowDriftTypeNForNuclear}, \cite{SummaryOfTCduringAGRExperiments}, \cite{ASTME320M}).


\section{Examples}
Eqn. \ref{exampleequation} is an equation. Fig.\ref{examplefigure} is a figure. Table \ref{exampletable} is a table.

\subsection{Subsection}
This is a subsection.

\subsubsection{Subsubsection}
This is a subsubsection. 

This is another paragraph

\begin{equation}\label{exampleequation}
E = mc^2
\end{equation}

\begin{table}[h]
\begin{center}
\caption{\label{exampletable}Table example.}
\begin{tabular}{ l c c c c}
\hline
\hline
                            &            ACRR               &           TREAT                  &         WSMR FBR           &         NIF  \\

\hline

Pulse length           &  $\approx$   &  $\approx$  &    $\approx$      &  $\approx$ \\
\hline
\hline
\end{tabular}
\end{center}
\end{table}


\begin{figure}[H]
\begin{center}
               \includegraphics[width=15cm,keepaspectratio]{report\_examplefig.png}
               \caption{\label{examplefigure}Figure example.}
\end{center}
\end{figure}












\newpage


% \bibliography{bibTemplate}

























\end{document}